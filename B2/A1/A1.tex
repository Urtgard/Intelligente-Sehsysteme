\section{ImageToolBox: Gamma-Korrektur}
\begin{enumerate}
	\setcounter{enumi}{1}
	\item Eine Gamma-Korrektur mit $\gamma=3$ dunkelt das Bild ab. \\
	      Der Verlauf der Korrekturfunktion zeigt, dass niedrige Intensitätswerte gestaucht werden: $[0,0.5] \rightarrow [0,0.2]$\\
	      Hohe Intensitätswerte werden gespreizt: $[0.8,1] \rightarrow [0.5,1]$ \\

	      Eine Gamma-Korrektur mit $\gamma=1$ hat keinen Effekt. \\
	      Der Verlauf der Korrekturfunktion zeigt, dass jeder Intensitätswert unverändert bleibt: $T_\gamma(I)=I$ \\

	      Eine Gamma-Korrektur mit $\gamma=0.3$ hellt das Bild auf. \\
	      Der Verlauf der Korrekturfunktion zeigt, dass niedrige Intensitätswerte gespreizt werden: $[0,0.2] \rightarrow [0,0.6]$\\
	      Hohe Intensitätswerte werden gestaucht: $[0.5,1] \rightarrow [0.8,1]$
\end{enumerate}

\begin{figure}
	\centering
	\begin{subfigure}{.32\textwidth}
		\centering
		\includegraphics[width=.99\linewidth]{A1/gamma=3.png}
		\caption{$\gamma=3$}
	\end{subfigure}
	\begin{subfigure}{.32\textwidth}
		\centering
		\includegraphics[width=.99\linewidth]{A1/gamma=1.png}
		\caption{$\gamma=1$}
	\end{subfigure}
	\begin{subfigure}{.32\textwidth}
		\centering
		\includegraphics[width=.99\linewidth]{A1/gamma=0.3.png}
		\caption{$\gamma=0.3$}
	\end{subfigure}
\end{figure}