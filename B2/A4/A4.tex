\section{Gauß-Filter}
\begin{enumerate}
	\setlength\itemsep{2em}
	\item $\begin{aligned}[t]
			      g(x,y) & = \frac{1}{2 \pi \sigma^2} \cdot e^{- \frac{x^2 + y^2}{2\sigma^2}} \\
			      \sigma & = 0.3                                                              \\
			      f(u,v) & =\frac{1}{\sum_u\sum_v g(u,v)} \begin{pmatrix}
				                                              g(-1,1)  & g(0,1)  & g(1,1)  \\
				                                              g(-1,0)  & g(0,0)  & g(1,0)  \\
				                                              g(-1,-1) & g(0,-1) & g(1,-1)
			                                              \end{pmatrix}        \\
			             & \approx \frac{1}{1.79587} \begin{pmatrix}
				                                         0.00003 & 0.00684 & 0.00003 \\
				                                         0.00684 & 1.76839 & 0.00684 \\
				                                         0.00003 & 0.00684 & 0.00003
			                                         \end{pmatrix}
		      \end{aligned}$
	\item Bei der Berechnung von $(f\ast g)(x,y)$ werden $\Theta(w_{kernel} \cdot h_{kernel})$ Operationen benötigt.
	      Für das gesamte Bild werden bei der Anwendung des normalen Filters
	      $\Theta(N \cdot w_{kernel} \cdot h_{kernel})$ Operationen benötigt.\\

	      Bei der Anwendung des separierten Filters werden einmal $\Theta(N \cdot w_{kernel})$
	      und einmal  $\Theta(N \cdot h_{kernel})$ Operationen benötigt.\\
	      Insgesamt werden also $\Theta(N (w_{kernel} + h_{kernel}))$ Operationen benötigt.

	\item Je größer $m$ gewählt wird, desto größer ist die Umgebung, die beim Filtern betrachtet wird.
	      Dementsprechend stärker ist die Glättung.\\
	      Der Einfluss auf die Glättung nimmt dabei mit steigendem Abstand zum Zentrum ab.
	      Bei $m=7$ betragen die Funktionswerte an den Ecken des Filterkerns nur 1 \% des Maximums der Gauß-Funktion.
	      Bei $m=5$ sind es dagegen noch fast 14 \%.



\end{enumerate}